{\large\textbf{Несобственный интеграл.}}

Мы требовали от интеграла огранниченость, теперь избавимся от этих ограничений. Если
функция кокается, например фукнция $f(x)=\cfrac{1}{\sqrt{x}}$ не опредена в 0. Но мы хотим
посчитать $\bigintss\limits_{0}^{1} f(x) dx$. Давайте это все определим через предел
в данном случае это будет просто $\lim\limits_{t\to0} \bigintss\limits_{t}^{1}f(x)dx $.
Аналогично поступим, если хотим посчитать $\bigintss\limits_{1}^{+\infty} \cfrac{1}{x^2}$

Из свойств - линейность, формула замены переменной, интегрирование по частям, если
функция меньше другой, то если интегралы сходятся, то интеграл меньшей функции будет меньше.

\textbf{ Признаки сходимости.}

1. Если $0 \leqslant f \leqslant g$, если $g$ - сходится, то $f$ сходится, если $f$
расходится, то и $g$ расходится.

2. Если $0 < g$ и существует $k \neq 0$ такое, что $ \lim\limits_{x \to b - 0} 
\cfrac{f(x)}{g(x)} = k$, то интегралы сходятся и расходятся одновременно.

\textbf{Критерий Коши.}  Пусть функция $f(x)$ определена на промежутке $[a ; b)$,
интегрируема в собственном смысле на любом отрезке $[a ; \xi], \xi<b$, и неограниченна 
в левой окрестности точки $x=b$. Тогда для сходимости интеграла
$ \bigintss\limits_{a}^{b} f(x) d x $
необходимо и достаточно, чтобы для любого числа $\varepsilon>0$ существовало такое
число $\eta \in[a ; b)$, что при любых $\eta_{1}, \eta_{2} \in(\eta ; b)$
$ \left|\bigintss\limits_{\eta_{1}}^{\eta_{2}} f(x) d x\right|<\varepsilon .  $

Обычно используется, для доказательства, что расходится.

\textbf{Абсолютная сходимость.} Вроде понятно что это. Если интеграл сходится
абсолютно, то он сходится.

\textbf{ Признак Дирихле.} Интеграл $\bigintss\limits_{a}^{b} f(x) g(x) d x$ сходится, если:\newline
a) функция $f(x)$ непрерывна и имеет ограниченную первообразную на $[a ; b)$;\newline
б) функция $g(x)$ непрерывно дифференцируема и монотонна на $[a ; b)$, причем $\lim\limits_{x \rightarrow b-0} g(x)=0 .$

\textbf{ Признак Абеля.} Интеграл $\bigintss\limits_{a}^{b} f(x) g(x) d x$ сходится, если:\newline
a) функция $f(x)$ непрерывна на $[a ; b)$ и интеграл $\bigintss\limits_{a}^{b} f(x) d x$ cxoдится;\newline
б) функция $g(x)$ ограниченна, непрерывно дифференцируема и монотонна на $[a ; b)$.


{\textbf{Выделение главной части.} } Идея простая. Мы заменяем функцию 
$f(x) = g(x) + R(x)$, где $R(x)$ - сходится абсолютно, а значит $f(x)$ и $g(x)$
сходятся и расходятся одновременно, а также абсолютно сходятся и расходятся одновременно.
Обычно работает как раз с границей $+\infty$(верю учебнику). Советую рассмотреть примеры
в учебнике это вторая часть параграф 12.

