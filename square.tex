\textbf{Площади плоских кривых и длин кривых} Тут просто формулы
которые стоит выучить, а дальше надо уметь считать интегральчики.  

Полощадь фигуры под графиком - просто интеграл. (иногда называют криволинейной
трапецией)

Если график задан параметрически(y(t), x(t)), а также $x'(t) \geqslant  0$, то 
$S = \bigintss\limits_{a}^{b} y(t)x'(t)dt$ - это просто замена переменной!

Пусть дана функция $r(\varphi )$ - это функция в полярных координатах, тогда
$S = \cfrac{1}{2} \bigintss\limits_{a}^{b} r^2(\varphi )d\varphi $

Длина кривой. $l = \bigintss\limits_{a}^{b} \sqrt{\sum\limits_{i=1}^{k} x'_i(t)} dt$.

Длина плоской кривой равна $l = \bigintss\limits_{a}^{b} \sqrt{1+y'^2}dx $.

Длина плоской кривой в полярных $l = \bigintss\limits_{a}^{b} \sqrt{r^2+r'^2}d\varphi $

\textbf{Объемы и площади тел вращения}. 

$V = \pi \bigintss\limits_{a}^{b} y^2(x)dx$

$V = \pi \bigintss\limits_{a}^{b} y^2(t)x'(t)dt$ - просто замена переменной. $x'\geqslant0$

Предыдущие две формулы легко запомнить, если думать, что твоя фукнция - это площадь круга
на одном слое, и ты просто берешь интеграл и получаешь объем.

$S = 2\pi \bigintss\limits_{a}^{b} |y(x)|\sqrt{1+y'^2(x)}dx$ - площадь фигуры вращения.

$S=2\pi \bigintss\limits_{a}^{b} y(t)\sqrt{x'^2(t)+ y'^2(t)}dt$ - площадь

$S = 2\pi\bigintss\limits_{a}^{b} r(\varphi )\sqrt{r^2(\varphi )+r'^2(\varphi )}d\varphi $
