{\large\textbf{Ряды.} }

Что такое ряд - понятно.

Необходимое свойство сходимости ряда $\sum\limits_{n=1}^{+\infty} a_n$ - это
$\lim\limits_{n \to \infty} a_n = 0$

\textbf{Критерий Коши.} $
\forall \varepsilon>0 \quad \exists N_{\varepsilon} \quad \forall n \geqslant N_{\varepsilon} \quad \forall p \in N:\left|a_{n+1}+a_{n+2}+\ldots+a_{n+p}\right|<\varepsilon
$

Критерий сходимости ряда с неотрицательными членами - это ограниченность частичных сумм
числом.

Признак сравнения рядов с неотр членами такой же как и у интегралов(ряд меньшей
последовательности меньше, а так же если существует передел отношения членов,
то ряды сходятся и расходятся одновременно).

\textbf{Интегральный признак.} Если функция $f$ не отрицательна и убивает, то 
ряд и интеграл сходятся одновременно

\textbf{Признак Даламбера.} Если для ряда
$
\sum\limits_{n=1}^{\infty} a_{n}, a_{n}>0 (n \in \mathbb{N})
$
существует такое число $0<q<1$, и таюой номер $n_{0}$, что для всех $n \geqslant n_{0}$ выполняется неравенство
$ a_{n+1} / a_{n} \leqslant q $
то этот ряд сходится; если же для всех $n \geqslant n_{0}$ имеет место неравенство
$ a_{n+1} / a_{n} \geqslant 1 $
то ряд расходится.

\textbf{Признак Коши.} Если для ряда
$ \sum\limits_{n=1}^{\infty} a_{n}, a_{n} \geqslant 0 (n \in \mathbb{N}) $
существует такое число $0 \leqslant q<1$, и такой номер $n_{0}$, что для всех $n \geqslant n_{0}$ выполняется неравенство
$ \sqrt[n]{a_{n}} \leqslant q $
то этот ряд сходится; если же для всех $n \geqslant n_{0}$ имеет место неравенство
$ \sqrt[n]{a_{n}} \geqslant 1 $ то ряд расходится.

\textbf{Абсолютно сходящиеся ряды.} Есть линейность, если переставить члены,
то будет сходиться к той же сумме.
