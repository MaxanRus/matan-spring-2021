{\large\textbf{Интегралы.}}


\textbf{Основные формулы} 

\begin{enumerate}

  \item $\bigintss 0 \cdot d x=C$
  \item $\bigintss d x=\bigintss 1 \cdot d x=x+C$
  \item $\bigintss x^{n} \cdot d x=\frac{x^{n+1}}{n+1}+C$, $n \neq-1, x>0$
  \item $\bigintss \frac{d x}{x}=\ln |x|+C$
  \item $\bigintss a^{x} d x=\frac{a^{x}}{\ln a}+C$
  \item $\bigintss e^{x} d x=e^{x}+C$
  \item $\bigintss \sin x d x=-\cos x+C$
  \item $\bigintss \cos x d x=\sin x+C$

  \item $\bigintss \frac{d x}{\sin ^{2} x}=-\operatorname{ctg} x+C$
  \item $\bigintss \frac{d x}{\cos ^{2} x}=\operatorname{tg} x+C$
  \item $\bigintss \frac{d x}{\sqrt{a^{2}-x^{2}}}=\arcsin \frac{x}{a}+C,|x|<|a|$
  \item $\bigintss \frac{d x}{a^{2}+x^{2}}=\frac{1}{a} \operatorname{arctg} \frac{x}{a}+C$
  \item «Высокий» логарифм:
    $\bigintss \frac{d x}{a^{2}-x^{2}}=\frac{1}{2 a} \ln \left|\frac{a+x}{a-x}\right|+C,|x| \neq a$
  \item «Длинный» логарифм:
    $\bigintss \frac{d x}{\sqrt{x^{2} \pm a^{2}}}=\ln \left|x+\sqrt{x^{2} \pm a^{2}}\right|+C$

\end{enumerate}

\textbf{Интегрирование с заменой переменной.} 

$\bigintss f(\varphi(x))\varphi'(x)dx=\bigintss f(t)dt$

Не забывайте заменять границы интегрирования, при замене переменной!

\textbf{Интегрирование по частям.} 

$\bigintss uv'dx = uv - \bigintss vu'dx$ или $\bigintss udv = uv - \bigintss vdu$

\textbf{Интегрирование простых дробей.} 

Мы легко умеем интегрировать дроби вида $\cfrac{1}{(x-a)}, \cfrac{1}{(x-a)^n}$. Немного
сложнее будет если захотим взять интеграл от $\cfrac{mx + n}{x^2+bx+c}$, где $b^2-4c < 0$.
Тут надо немного подумать и привести к виду $\cfrac{1}{(x+a)^2+b}+\cfrac{x+a}{(x+a)^2+b}$.
Далее в первом вносим в дифф 1, чтобы получить $d(x+a)$, а это формула 12. Во втором
вносим $x + a$ под дифф и получим $\bigintss \cfrac{1}{t+b}dt$ - это тоже умеем.\newline
Если у нас какая-то большая дробь, то раскладываем знаменатель на множители 
$(x-x_1), (x-x_2), \dots, (x-x_n)$ и $(x^2 + p_1x+q_k), \dots, (x^k + p_kx + q_k)$.
Далее просто приводим нашу дробь к сумме дробей $\cfrac{A_1}{x-x_1} + \cfrac{A_2}{x-x_2} +
\dots + \cfrac{B_1x+C_1}{x^2 + p_1x + q_1} + \dots $. Это можно посчитать с помощью
СЛУ(если записать СЛУ, то становится очевидно, что есть решение).

\textbf{Интегрирование иррациональных функций} Я этой штуки рот мылом мыл.

\textbf{Интегрирование трансцендентных функций} Тут не много отличий от предыдущего.
